\chapter{Conclusion}
In this work, it is found that the daytime satellite images can help to predict wealth distributions resulting in more accurate results as expected. As discussed in this paper, due to the data gaps only one source of data sets would be insufficient to predict better results. Both daytime and nighttime imagery is thus used. The household recode data has been used for verification. Another feature of this model is that it can perform on any resolution of satellite images but usage of high resolution images is recommended to achieve desired outputs. It is a cheaper, high performing and reliable source to fill the data gaps from the survey data using the widely available satellite images. It is not feasible for regions with very little or no survey data to use a similar approach as the model is highly dependable on previous data for training and predictions. Ridge regression models are used for this experiment because of its lesser time consumption and greater $R^2$ scores when compared with Lasso and Linear regressions. To compare the results obtained with existing models, there were no experiments performed on all the countries chosen for this work.  


%\section{Discussion}

\section{Future Work}
For the future work, we plan on using the models trained on one region with the other region's data to observe how they perform and also try to observe the performance of the models when trained on multiple regions at once.

\section{Final Words}
These predictions can help with research and analysis that can help many organizations and governments make better decisions on where to devote their time and resources efficiently in order to help enhance the economical standards of particular undeveloped regions.


