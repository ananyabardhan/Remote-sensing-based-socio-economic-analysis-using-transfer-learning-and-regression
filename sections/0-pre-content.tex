\newpage
\thispagestyle{plain}
~\\
\vfill
{ \setstretch{1.1}
	\subsection*{Authors}
	Sree Teja Buddaraju  \\ Ananya Bardhan \\
	Ramya Sri Boddu \\
	Simranjit Kaur \\

	\subsection*{Supervisor}
	Dr. Thangarajah Akilan\\
% 	Thunder Bay, Ontario\\
% 	Lakehead University
	
	\subsection*{Place for Project}
	Department of Computer Science\\
	Lakehead University\\
	955 Oliver Rd, Thunder Bay, ON P7B 5E1\\
	Canada
	~
%	\subsection*{Examiner}
%	The Professor
%	Place \\
%	KTH Royal Institute of Technology
	

}


\newpage
\thispagestyle{plain}
%%%%%%%%%%%%%%%%%%%%%%%%%%%%%%%%%%%%
%%  The English abstract          %%
%%%%%%%%%%%%%%%%%%%%%%%%%%%%%%%%%%%%
\chapter*{Abstract}
%%%%%%%%%%%%%%%%%%%%%%%%%%%%%%%%%%%%


\vspace{0.5cm}
The economic status of each country varies. Some countries are well developed while some are underdeveloped. A lower economic status in any state can lead to hunger, malnutrition, and low life expectancy of the people living there, especially for children and the older generation. Most of them live below the international poverty line of 1.25 US dollar per day according to the World Bank Group. One way of working towards solving this problem is through collecting information for analysis which works as a vital resource for the organizations that can help the people there to lead a better life. But obtaining this information in the conventional way through human surveys takes too long and requires a lot of resources. This work is intended to act as an alternative to this problem, it analyzes the social and economic status of the underdeveloped countries, primarily the selected African countries, by exploiting machine learning and image processing technologies. This work proposes a set of algorithms that can make predictions on the standard of living of a particular geographic region based on the distribution of night lights observed through remote sensing and satellite image processing to provide more accurate results. This work is based on transfer learning where we train a model pre-trained on ImageNet and obtain predictions of the wealth distribution of a region using the data from National Oceanic And Atmospheric Administration (NOAA), Demographic and Health Survey (DHS), and Google static maps. The results show the performance analysis of the the models built, based on the correlation between the actual wealth as observed in the survey data and the wealth index predicted by the model. The predicted wealth is mapped on a heat map for visualizing the wealth distribution of the particular regions.




\subsection*{Keywords}
Remote sensing, Socioeconomic analysis, Machine learning







\newpage
\thispagestyle{plain}
\chapter*{Acknowledgements}
We would like to share  our sincere gratitude towards professor Dr. Thangarajan Akilan for actively supporting the idea behind this project and for his guidance and advice throughout its duration. We would like to thank our colleagues and peers who provided subjective measures to improve this project. 



\newpage
\chapter*{Acronyms}

\begin{acronym}[RDBMS]
%\acro{ACID}{atomicity, consistency, isolation, and durability}
%\acro{CAP}{Consistency, Availability, Partition-tolerant}
%\acro{CDF}{Cumulative Distribution Function}
\acro{DHS}{Demographic Health Survey}
\acro{NOAA}{National Oceanic and Atmospheric Administration}
\acro{NGDC}{National Geophysical Data Center}
\acro{EOG}{Earth Observation Group}
\acro{GDP}{Gross Domestic Product}
\acro{GDAL}{Geospatial Data Abstraction Library}
\acro{NGO}{Non Government Organization}
\acro{API}{Application Programming Interface}
\acro{CNN}{Convolution Neural Network}
\acro{PCA}{Principal Component Analysis}
\acro{MSE}{Mean Square Error}
\end{acronym}


\etocdepthtag.toc{mtchapter}
\etocsettagdepth{mtchapter}{subsection}
\etocsettagdepth{mtappendix}{none}
\thispagestyle{plain}
\tableofcontents

\newpage


